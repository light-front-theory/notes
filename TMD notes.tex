\documentclass[a4paper,12pt]{article}
\setlength{\textwidth}{18cm}
\setlength{\topmargin}{-3cm}
\setlength{\textheight}{24cm}
\setlength{\oddsidemargin}{-1cm}
\setlength{\evensidemargin}{-7cm}
\setlength{\headheight}{2.5cm}
\setlength{\topskip}{-6cm}
\linespread{1.5}
\usepackage{CJKutf8}
\usepackage{comment}
\usepackage{amssymb}
\usepackage{color}
\usepackage{amsmath}
\usepackage{ulem}
\usepackage{graphicx}
\usepackage{dcolumn}
\usepackage{bm}
\usepackage{slashed}
\usepackage{indentfirst}
\usepackage{booktabs}
\usepackage{amssymb,amsfonts}
\usepackage{amsmath}
\usepackage{color}
\usepackage{array}
\usepackage{subfigure}
\usepackage{cases}
\usepackage[bookmarks]{hyperref}
\usepackage[center]{titlesec}
\begin{document}
\title{Transverse Momentum Distribution Function notes}
\author{Siqi-Xu}
\maketitle
%%%%%%%%%%%%%%%%%%%%%%%%%%%%%%%%%%%%%%%%%%%%%%%%%%%%%%%%%%%
\section{Introduction}

In this paper, we mainly display the process about the calculation of the leading twist
 transvarse-momentum dependent distribution function.
For simplicity, we only introduce the unpolarized internal electron distribution function,
 and there are two functions one of which is $f_1^e$ and another one is Sivers Function
 $f_{1T}^{\perp e}$.
As far as the notation is concerned, the T refers to the spin of the dressed electron;
 the "$\perp$" symbols signals an explicit dependence on transverse momenta with an
 uncontracted index.
Also, we try to show the behaviour of the distribution function in the time-reverse and
 the distribution function product effect on the single spin asymmetry.

\section{The transformation of the basis}

According to the Ref.~\cite{prd1}, we can get the correlation function:
\begin{eqnarray}
\Phi^{[\gamma^+]}(x,k_{\perp};S)&=& f_1^e -\frac{\epsilon^{ij}_{\perp}k^i_{\perp}S^i_{\perp}}{m}f_{1T}^{\perp e}\label{eq1.1}\\
\Phi^{[\gamma^+]}(x,k_{\perp};S)&=& \frac{1}{2}Tr[\Phi(x,k_{\perp};S)\gamma^+]\label{eq1.2}\\
\Phi^{[\gamma^+]}(x,k_{\perp};S)&=& \int \frac{d \xi^- d^2 \xi^{\perp}}{2(2\pi)^3} e^{ik \cdot \xi} <P,S|\bar{\psi}(0)\gamma^+\psi(\xi)|P,S>\label{eq1.3}\
\end{eqnarray}

And now, we can find the basis $|P,S>$ correspoending to the momentum and the spin of the
 physics particle. But if we want to calculate the function by BLFQ, we need to
 decompose the physics particle state in terms of the light cone helicity state $|P,\Lambda>$.
 Over here, we employ the SU(2) rotation to decompose the state.
 If we assume $S=(sin\theta cos \phi, sin\theta sin \phi, cos \theta)$, we need to transformation the state $|P,S>$ according to:
\begin{eqnarray}
\begin{pmatrix}
|P,S> \\ |P,-S>
\end{pmatrix}\quad
&=& u(\theta, \phi)
\begin{pmatrix}
|P,\Lambda> \\ |P,\Lambda>
\end{pmatrix}\\
u(\theta ,\phi) &=& \begin{pmatrix}
cos \frac{\theta}{2} e^{-i\frac{\phi}{2}} & sin \frac{\theta}{2} e^{-i\frac{\phi}{2}} \\
-sin \frac{\theta}{2} e^{i\frac{\phi}{2}} & cos \frac{\theta}{2} e^{i\frac{\phi}{2}}
\end{pmatrix}\
\end{eqnarray}

where the $u(\theta, \phi)$ is the SU(2) rotation matrix.
Now, let us consider the case of the transverse-momentum dependent correlator function define in Eq.(~\ref{eq1.3}).
We can choice a reference frame where the incoming dressd electron has a momentum $P=(P^+,P^-,0)$.
So the parameter can be fixed at $\theta = \frac{\pi}{2}, \phi = 0$(Be careful, that choose is not necessary. But it can make the physical significance more explicit), and the rotation matrix can be given by:
\begin{eqnarray}
u(\theta,\phi)=\begin{pmatrix}
\frac{\sqrt{2}}{2} & \frac{\sqrt{2}}{2} \\ -\frac{\sqrt{2}}{2} & \frac{\sqrt{2}}{2}
\end{pmatrix}\label{eq1.5}\
\end{eqnarray}

From now on, we can rewrite the correlator function Eq.(~\ref{eq1.3}) as:
\begin{eqnarray}
\Phi^{[\gamma^+]}(x,k_{\perp};+\frac{1}{2})&=& \int \frac{d \xi^- d^2 \xi^{\perp}}{2(2\pi)^3} e^{ik \cdot \xi} [<P,\Lambda|\bar{\psi}(0)\gamma^+\psi(\xi)|P,\Lambda>+<P,\Lambda|\bar{\psi}(0)\gamma^+\psi(\xi)|P,-\Lambda>\nonumber\\&&+<P,-\Lambda|\bar{\psi}(0)\gamma^+\psi(\xi)|P,\Lambda>+<P,-\Lambda|\bar{\psi}(0)\gamma^+\psi(\xi)|P,-\Lambda>]\nonumber\\
\Phi^{[\gamma^+]}(x,k_{\perp};-\frac{1}{2}) &=& \int \frac{d \xi^- d^2 \xi^{\perp}}{2(2\pi)^3} e^{ik \cdot \xi} [<P,\Lambda|\bar{\psi}(0)\gamma^+\psi(\xi)|P,\Lambda>-<P,\Lambda|\bar{\psi}(0)\gamma^+\psi(\xi)|P,-\Lambda>\nonumber\\&&-<P,-\Lambda|\bar{\psi}(0)\gamma^+\psi(\xi)|P,\Lambda>+<P,-\Lambda|\bar{\psi}(0)\gamma^+\psi(\xi)|P,-\Lambda>]\label{eq1.6}\
\end{eqnarray}

And then, if we introduce the correlator function into Eq.(~\ref{eq1.1}) and Eq.(~\ref{eq1.2}), we find:
\begin{eqnarray}
2f_1^e &=& \frac{1}{2} Tr[(\Phi(x,k_T;+\frac{1}{2})+\Phi(x,k_T;-\frac{1}{2}))\gamma^+]\nonumber\\
&=&\int \frac{d \xi^- d^2 \xi^{\perp}}{2(2\pi)^3} e^{ik \cdot \xi} (<P,\Lambda|\bar{\psi}(0)\gamma^+\psi(\xi)|P,\Lambda>+<P,-\Lambda|\bar{\psi}(0)\gamma^+\psi(\xi)|P,-\Lambda>)\label{eq1.7}\\
-2\frac{\epsilon^{ij}_{\perp}k_{\perp}^i S^j_{\perp}}{m}f^{\perp e}_{1T}&=& \frac{1}{2}Tr[(\Phi(x,k_{T};+\frac{1}{2})-\Phi(x,k_{T};-\frac{1}{2}))\gamma^+]\nonumber\\
&=& \int \frac{d \xi^- d^2 \xi^{\perp}}{2(2\pi)^3} e^{ik \cdot \xi}(<P,-\Lambda|\bar{\psi}(0)\gamma^+\psi(\xi)|P,\Lambda>+<P,\Lambda|\bar{\psi}(0)\gamma^+\psi(\xi)|P,-\Lambda>)\label{eq1.8}\
\end{eqnarray}

where the result was show in Ref.~\cite{prd2}.

\section{Deduce the correlator function in helicity basis}

In this chapter, we mainly deduce the correlator function from the Eq.(~\ref{eq1.7}) and Eq.(~\ref{eq1.8}).
Firstly, we use the perturbative theory show in Ref.~\cite{prd1} to get a result about $f_1^e$.
Secondly, we use BLFQ to get the result include $f_1^e$ distribution function and $f_{1T}^{\perp e}$ distribution function.
It is worth mentioning that we now adopted the light-cone gauge without the gauge link. And we will introduce the gauge link at next chapter.
For the sake of convenient reference, we defined a correlator function as:
\begin{eqnarray}
\Phi^{\gamma^+}(x,k_{\perp};\Lambda,\Lambda^{\prime}) = \int \frac{d \xi^- d^2 \xi^{\perp}}{2(2\pi)^3} e^{ik \cdot \xi}<P,\Lambda|\bar{\psi}(0)\gamma^+\psi(\xi)|P,\Lambda^{\prime}>\label{eq1.9}
\end{eqnarray}

\subsection{The result of perturbative theory}

According to Ref.~\cite{prd1}, we mainly consider the dressed electron.
Therefore, at this stage we are allowed to calculation that for $x\neq1, k_{\perp}\neq 0_{\perp}$.ib
In other word, we can expand the physics state as:
\begin{eqnarray}
|P,\Lambda> &=&\sqrt{z} |e> + |e\gamma>\nonumber\\
|e> &=& b^+_{\Lambda}(P)|0>\nonumber\\
|e\gamma> &=& \int \frac{dxd^2k_{\perp}}{2(2\pi)^3\sqrt{x(1-x)}} \sum_{\lambda,\lambda_{\gamma}} \Psi^{\Lambda}_{\gamma,\gamma_{\lambda}} |e\gamma;p_e,\lambda;p_{\gamma},\lambda_{\gamma}>\
\end{eqnarray}

And then, we obtain the LFWF overlap representation of the correlator, which reads:
\begin{eqnarray}
\Phi^{\gamma^+}(x,k_{\perp};\Lambda,\Lambda^{\prime}) &=& \int \frac{d\xi^- d^2\xi^{\perp}}{2(2\pi)^3}e^{ik \cdot \xi} \int \frac{dx_1d^2p_{1e\perp}}{2(2\pi)^3\sqrt{x_1(1-x_1)}} \sum_{\lambda_1,\lambda_{1\gamma}}  \int \frac{dx_2d^2p_{2e\perp}}{2(2\pi)^3\sqrt{x_2(1-x_2)}} \sum_{\lambda_2,\lambda_{2\gamma}}\nonumber\\
&& <e\gamma;p_{1e},\lambda_1,p_{1\gamma},\lambda_{1\gamma}|\Psi^{\Lambda \star}_{\lambda_1,\lambda_{1\gamma}}\bar{\Psi}(0)\gamma^+\Psi(\xi)\Psi^{\Lambda^{\prime}}_{\lambda_2,\lambda_{2\gamma}}|e\gamma;p_{2e},\lambda_2,p_{2\gamma},\lambda_{2\gamma}>\nonumber\\
&=& \int \frac{d\xi^- d^2\xi^{\perp}}{2(2\pi)^3}e^{ik \cdot \xi} \int \frac{dx_1d^2p_{1e\perp}}{2(2\pi)^3\sqrt{x_1(1-x_1)}} \sum_{\lambda_1,\lambda_{1\gamma}}  \int \frac{dx_2d^2p_{2e\perp}}{2(2\pi)^3\sqrt{x_2(1-x_2)}} \sum_{\lambda_2,\lambda_{2\gamma}}\nonumber\\
&&\int \frac{d p_1^+d^2p_{1\perp}}{(2\pi)^3 2p_1^+} \int \frac{d p_2^+ d^2 p_{2\perp}}{(2\pi)^3 2p_2^+} \sum_{\lambda} \sum_{\lambda^{\prime}}<e\gamma;p_{1e},\lambda_1;p_{1\gamma},\lambda_{1\gamma}|\Psi^{\Lambda \star}_{\lambda_1,\lambda_{1\gamma}}\nonumber\\
&&[b^+(p_1,\lambda)\bar{u}(p_1,\lambda)+d(p_1,\lambda)\bar{v}(p_1,\lambda)]\gamma^+[b(p_2,\lambda^{\prime})u(p_2,\lambda^{\prime})e^{-ip_2\cdot \xi}+d^+(p_2,\lambda^{\prime})v(p_2,\lambda^{\prime})e^{ip_2 \cdot \xi}]\nonumber\\
&&\Psi^{\Lambda^{\prime}}_{\lambda_2,\lambda_{2\gamma}}|e\gamma;p_{2e},\lambda_2;p_{2\gamma},\lambda_{2\gamma}>\nonumber\\
&=&\int \frac{d\xi^- d^2\xi^{\perp}}{2(2\pi)^3}e^{ik \cdot \xi} \int \frac{dx_1d^2p_{1e\perp}}{2(2\pi)^3\sqrt{x_1(1-x_1)}} \sum_{\lambda_1,\lambda_{1\gamma}}  \int \frac{dx_2d^2p_{2e\perp}}{2(2\pi)^3\sqrt{x_2(1-x_2)}} \sum_{\lambda_2,\lambda_{2\gamma}}\nonumber\\
&&\int \frac{d p_1^+d^2p_{1\perp}}{(2\pi)^3 2p_1^+} \int \frac{d p_2^+ d^2 p_{2\perp}}{(2\pi)^3 2p_2^+} \sum_{\lambda} \sum_{\lambda^{\prime}}<0|\Psi^{\Lambda \star}_{\lambda_1,\lambda_{1\gamma}}\bar{u}(p_1,\lambda)\gamma^+u(p_2,\lambda^{\prime})e^{-ip_2\cdot \xi}\Psi^{\Lambda^{\prime} }_{\lambda_2,\lambda_{2\gamma}}|0>\nonumber\\
&&2(2\pi)^3 p_{\gamma}^+\delta(p_{1\gamma}-p_{2\gamma}) 2(2\pi)^3 p_{1e}^+\delta(p_1-p_{1e}) 2(2\pi)^3 p_{2e}^+\delta(p_2-p_{2e})\delta^{\lambda}_{\lambda_1}\delta^{\lambda^{\prime}}_{\lambda_2}\delta^{\lambda_{1\gamma}}_{\lambda_{2\gamma}}\nonumber\\
&=&\int \frac{d\xi^- d^2\xi^{\perp}}{2(2\pi)^3}e^{ik \cdot \xi} \int \frac{dxd^2p_{e\perp}}{2(2\pi)^3 x p^+} \sum_{\lambda_1,\lambda_2,\lambda_{\gamma}}e^{-ip_e \cdot \xi}\nonumber\\
&&<0|\Psi^{\Lambda \star}_{\lambda_1,\lambda_{\gamma}}\bar{u}(p_e,\lambda_1)\gamma^+u(p_e,\lambda_2)\Psi^{\Lambda^{\prime}}_{\lambda_2,\lambda_{\gamma}}|0>\label{eq3.2}
\end{eqnarray}

For the dressed electron, we can get the expression:
\begin{eqnarray}
\int \frac{d\xi^- d^2\xi^{\perp}}{(2\pi)^3}e^{ik \cdot \xi} e^{-ip_e \cdot \xi} &=& \int \frac{d\xi^- d^2\xi^{\perp}}{(2\pi)^3} e^{i(k-p_e) \cdot \xi}\nonumber\\
&=& 2\delta(k^+-p_e^+)\delta^2(k_{\perp}-p_{e\perp})\label{eq3.3}
\end{eqnarray}

And then, it's easy to get a expression:
\begin{eqnarray}
\Phi^{\gamma^+}(x,k_{\perp};\Lambda,\Lambda^{\prime}) &=& \frac{1}{2(2\pi)^3 xP^+} \sum_{\lambda_1,\lambda_2,\lambda_{\gamma}} \Psi^{\Lambda \star}_{\lambda_1,\lambda_{\gamma}}\bar{u}(k_e,\lambda_1)\gamma^+u(k_e,\lambda_2)\Psi^{\Lambda^{\prime}}_{\lambda_2,\lambda_{\gamma}}\nonumber\\
&=& \frac{1}{2(2\pi)^3} \sum_{\lambda,\lambda_{\gamma}} \Psi^{\Lambda \star}_{\lambda,\lambda_{\gamma}}\Psi^{\Lambda^{\prime}}_{\lambda,\lambda_{\gamma}}
\end{eqnarray}

So we can get the distribution function:
\begin{eqnarray}
f_1^e &=& \frac{1}{4(2\pi)^3} \sum_{\Lambda,\lambda,\lambda_{\gamma}} \Psi^{\Lambda \star}_{\lambda,\lambda_{\gamma}}(p_e,p_{\gamma})\Psi^{\Lambda}_{\lambda,\lambda_{\gamma}}(p_e,p_{\gamma})\\
-\frac{\epsilon^{ij}_{\perp}k_{\perp}^i S^j_{\perp}}{m}f^{\perp e}_{1T} &=& \frac{1}{4(2\pi)^3} \sum_{\Lambda,\lambda,\lambda_{\gamma}} \Psi^{\Lambda \star}_{\lambda,\lambda_{\gamma}}(p_e,p_{\gamma})\Psi^{-\Lambda}_{\lambda,\lambda_{\gamma}}(p_e,p_{\gamma})
\end{eqnarray}

\subsection{The result of BLFQ without gauge link}

In this sebsection, we mainly introduce the BLFQ into the calculation of TMD without any gauge link. And we expect to get the distributation functions similar to the consequence which we show in previous subsection.

the most general expression of the expand of physics state ia given by:
\begin{eqnarray}
  |e_{phys}> &=& \psi_1|e> + \psi_2|e\gamma> + ......\nonumber\\
  |P,\Lambda> &=& \sum_n \int \prod_{i=1}^{n} \frac{dx_i d^2k_{\perp i}}{\sqrt{x_i}16\pi^3} 16\pi^3\delta(1-\sum_{i=1}^{n}x_i) \delta^2(\sum_{i=1}^{n} k_{\perp i}-P_{\perp}) \Psi_n(x_i, k_{\perp i};\lambda_i)|n,x_i,x_iP^+,k_{\perp i},\lambda_i>\nonumber
\end{eqnarray}

\subsubsection{The basis of BLFQ}

Over here, we only consider two Fock sector $|e>$ and $|e\gamma>$. According to the basis of BLFQ, we need to discrete the momentum of longitudinal direction as:
\begin{numcases}{P^+ = \frac{2\pi n}{L}}
  n=1,2,3,\cdots\cdots periodic\ boundary\ condition\ for\ boson\nonumber\\
  n=\frac{1}{2},\frac{3}{2},\frac{5}{2},\cdots\cdots antiperiodic\ boundary\ conditon\ for\ fermion\nonumber\
\end{numcases}

And we use the 2D-HO to replace the transverse momentum basis, like:
\begin{eqnarray}
  a^+ (p,\lambda) &=& a^+_{\alpha} \Phi_{n,m}(p_{\perp}), \ \ \  \alpha=\{k^+,n,m, \lambda\}\nonumber\\
  \{a(P,\Lambda),a^+(P,\Lambda)\} &=& (2\pi)^2 \delta^2(P_{\perp}-P^{\prime}_{\perp}) \delta_{P^+}^{P^{\prime +}} \nonumber\\
  \{a_{\alpha},a^+_{\bar{\alpha}}\} &=& \delta^{\alpha}_{\bar{\alpha}}\nonumber\
\end{eqnarray}

From now on, we mainly introduce two process to get the final expression, one of which use the state $|p,\lambda>$ in the momentum space act on the operator $\bar{\Psi} \gamma^+ \Psi$ and another one directly use our basis to expand the physics state, at the same time act on the operator $\bar{\Psi} \gamma^+ \Psi$.

\subsubsection{discuss in momentum space}

According to the properties of projection operator, we can begin with Eq.(~\ref{eq1.9}) and get the general form of the expression. Firstly, we expand the physics state by our basis and momentum basis:
\begin{eqnarray}
  &&<P_{\perp},\alpha| \bar{\Psi}(0) \gamma^+ \Psi(\xi) |P^{\prime}_{\perp}, \alpha^{\prime}> =\sum_{\alpha_1,\alpha_2,a_1,a_2}\int \frac{d^2 p_{1\perp}}{(2\pi)^2} \int \frac{d^2 p_{2\perp}}{(2\pi)^2}\nonumber\\&& \times <P_{\perp},\alpha_1|p_{1\perp},a_1><p_{1\perp},a_1|\bar{\Psi}(0) \gamma^+ \Psi(\xi) |p_{2\perp},a_2><p_{2\perp},a_2|P_{\perp},\alpha_2>\nonumber\\
  &&+\sum_{\alpha_1,\alpha_2,a_{1e},a_{1\gamma},a_{2e},a_{2\gamma}} \int \frac{d^2 p_{1e\perp}}{(2\pi)^2}\int \frac{d^2 p_{1\gamma\perp}}{(2\pi)^2}\int \frac{d^2 p_{2e\perp}}{(2\pi)^2}\int \frac{d^2 p_{2\gamma\perp}}{(2\pi)^2} <P_{\perp},\alpha_1|p_{1e\perp},a_{1e},p_{1\gamma\perp},a_{1\gamma}>\nonumber\\&&<p_{1e\perp},a_{1e},p_{1\gamma\perp},a_{1\gamma}| \bar{\Psi}(0) \gamma^+ \Psi(\xi) |p_{2e\perp},a_{2e},p_{2\gamma\perp},a_{2\gamma}><p_{2e\perp},a_{2e},p_{2\gamma\perp},a_{1\gamma}|P_{\perp},\alpha_2>\label{eq3.2.1}
\end{eqnarray}

where $a=\{p^+,\lambda\}$ and the convention is shown in Appendix 4.1.2. And then we let the state in momentum space act on the operator:
\begin{eqnarray}
  <p_{1\perp},a_1|\bar{\Psi}(0) \gamma^+ \Psi(\xi) |p_{2\perp},a_2> &&=\sum_{a_3, a_4} \frac{1}{2L} \int \frac{d^2p_3}{(2\pi)^2} \int \frac{d^2p_4}{(2\pi)^2} e^{-ip_4 \cdot \xi} \nonumber\\
  &&<0| b(p_{1\perp},a_1) b^+(p_{3\perp},a_3) \bar{u}(p_{3\perp},a_3) \gamma^+ u(p_{4\perp},a_4) b(p_{4\perp},a_4) b^+(p_{2\perp},a_2)|0>\nonumber\\
  &&= \sum_{a_3, a_4} \frac{1}{2L}\int \frac{d^2p_3}{(2\pi)^2} \int \frac{d^2p_4}{(2\pi)^2} e^{-ip_4 \cdot \xi} \nonumber\\
  && <0|  \bar{u}(p_{3\perp},a_3) \gamma^+ u(p_{4\perp},a_4) |0> (2\pi)^2\delta^2(p_{1\perp}-p_{3\perp}) (2\pi)^2\delta^2(p_{2\perp}-p_{4\perp})\delta^{a_1}_{a_3} \delta^{a_2}_{a_4}\nonumber\\
  &&= \frac{1}{2L}\bar{u} (p_{1\perp},a_1) \gamma^+ u(p_{2\perp},a_2) e^{-ip_2 \cdot \xi}\label{eq3.2.2}
\end{eqnarray}
\begin{eqnarray}
  &&<p_{1e\perp},a_{1e},p_{1\gamma\perp},a_{1\gamma}| \bar{\Psi}(0) \gamma^+ \Psi(\xi) |p_{2e\perp},a_{2e},p_{2\gamma\perp},a_{2\gamma}>\nonumber\\
  &&=\sum_{a_3, a_4} \frac{1}{2L}\int \frac{d^2p_3}{(2\pi)^2} \int \frac{d^2p_4}{(2\pi)^2} e^{-ip_4 \cdot \xi}\nonumber\\&& <0| b(p_{1e\perp},a_{1e})a(p_{1\gamma\perp},a_{1\gamma}) b^+(p_{3\perp},a_3) \bar{u}(p_{3\perp},a_3) \gamma^+ u(p_{4\perp},a_4) b(p_{4\perp},a_4) b^+(p_{2e\perp},a_{2e})a^+(p_{2\gamma\perp},a_{2\gamma})|0>\nonumber\\
  &&= \sum_{a_3, a_4} \frac{1}{2L}\int \frac{d^2p_3}{(2\pi)^2} \int \frac{d^2p_4}{(2\pi)^2} e^{-ip_4 \cdot \xi}\nonumber\\&& <0| \bar{u}(p_{3\perp},a_3) \gamma^+ u(p_{4\perp},a_4)|0> (2\pi)^2 \delta^2(p_{1e\perp}-p_{3\perp})\delta^{a_{1e}}_{a_3}(2\pi)^2 \delta^2(p_{4\perp}-p_{2e\perp})\delta^{a_{4}}_{a_{2e}}(2\pi)^2 \delta^2(p_{1\gamma\perp}-p_{2\gamma\perp})\delta^{a_{1\gamma}}_{a_{2\gamma}}\nonumber\\
  &&= \frac{1}{2L}\bar{u}(p_{1e\perp},a_{1e}) \gamma^+ u(p_{2e\perp},a_{2e}) e^{-ip_{2e} \cdot \xi}(2\pi)^2 \delta^2(p_{1\gamma\perp}-p_{2\gamma\perp})\delta^{a_{1\gamma}}_{a_{2\gamma}}\label{eq3.2.3}
\end{eqnarray}

Then we can introduce the final result of Eq.~\ref{eq3.2.2} and Eq.~\ref{eq3.2.3} into the Eq.~\ref{eq3.2.1} :
\begin{eqnarray}
  &&<P_{\perp}=0,\alpha| \bar{\Psi}(0) \gamma^+ \Psi(\xi) |P_{\perp}=0, \alpha^{\prime}>\nonumber\\ &&= \sum_{\alpha_1,\alpha_2,a_1,a_2}\int \frac{d^2 p_{1\perp}}{(2\pi)^2} \int \frac{d^2 p_{2\perp}}{(2\pi)^2}\nonumber\\&& \times <P_{\perp}=0,\alpha_1|p_{1\perp},a_1>\frac{1}{2L}\bar{u} (p_{1\perp},a_1) \gamma^+ u(p_{2\perp},a_2) e^{-ip_2 \cdot \xi}<p_{2\perp},a_2|P_{\perp}=0,\alpha_2>\nonumber\\
  &&+\sum_{\alpha_1,\alpha_2,a_{1e},a_{1\gamma},a_{2e},a_{2\gamma}} \int \frac{d^2 p_{1e\perp}}{(2\pi)^2}\int \frac{d^2 p_{1\gamma\perp}}{(2\pi)^2}\int \frac{d^2 p_{2e\perp}}{(2\pi)^2}\int \frac{d^2 p_{2\gamma\perp}}{(2\pi)^2} <P_{\perp}=0,\alpha_1|p_{1e\perp},a_{1e},p_{1\gamma\perp},a_{1\gamma}>\nonumber\\&&\frac{1}{2L}\bar{u}(p_{1e\perp},a_{1e}) \gamma^+ u(p_{2e\perp},a_{2e}) e^{-ip_{2e} \cdot \xi}(2\pi)^2 \delta^2(p_{1\gamma\perp}-p_{2\gamma\perp})\delta^{a_{1\gamma}}_{a_{2\gamma}}<p_{2e\perp},a_{2e},p_{2\gamma\perp},a_{1\gamma}|P_{\perp}=0,\alpha_2>\nonumber\\
  &&=\sum_{\alpha_1,\alpha_2,a_1,a_2}\int \frac{d^2 p_{1\perp}}{(2\pi)^2} \int \frac{d^2 p_{2\perp}}{(2\pi)^2}\nonumber\\&& \times <P_{\perp}=0,\alpha_1|p_{1\perp},a_1>\frac{1}{2L}\bar{u} (p_{1\perp},a_1) \gamma^+ u(p_{2\perp},a_2) e^{-ip_2 \cdot \xi}<p_{2\perp},a_2|P_{\perp}=0,\alpha_2>\nonumber\\
  &&+\sum_{\alpha_1,\alpha_2,a_{1e},a_{1\gamma},a_{2e}} \int \frac{d^2 p_{1e\perp}}{(2\pi)^2}\int \frac{d^2 p_{1\gamma\perp}}{(2\pi)^2}\int \frac{d^2 p_{2e\perp}}{(2\pi)^2} <P_{\perp}=0,\alpha_1|p_{1e\perp},a_{1e},p_{1\gamma\perp},a_{1\gamma}>\nonumber\\&&\frac{1}{2L}\bar{u}(p_{1e\perp},a_{1e}) \gamma^+ u(p_{2e\perp},a_{2e}) e^{-ip_{2e} \cdot \xi}<p_{2e\perp},a_{2e},p_{1\gamma\perp},a_{1\gamma}|P_{\perp}=0,\alpha_2>\nonumber\
\end{eqnarray}

Over here, we can see whatever we do we don't use the normalization condition. And starting form the physical insight and mathematica, we introduce the 2D-HO basis so that  the tranverse momentum is not zero. So we need to use the more general form of the operator.
\begin{eqnarray}
  &&\int \frac{d^4x}{(2\pi)^4} e^{-iP_c \cdot x} \int \frac{d\xi^- d^2 \xi^{\perp}}{(2\pi)^3} e^{i(k+P_c)\cdot(x+\xi)} <P_{\perp}=0,\alpha| \bar{\Psi}(x) \gamma^+ \Psi(x+\xi) |P_{\perp}=0, \alpha^{\prime}>\nonumber\\
  &&=\int \frac{d^4x}{(2\pi)^4} e^{-iP_c \cdot x} \int \frac{d\xi^- d^2 \xi^{\perp}}{(2\pi)^3} e^{i(k+P_c)\cdot(x+\xi)}[\sum_{\alpha_1,\alpha_2,a_1,a_2}\int \frac{d^2 p_{1\perp}}{(2\pi)^2} \int \frac{d^2 p_{2\perp}}{(2\pi)^2}\nonumber\\&& \times <P_{\perp}=0,\alpha_1|p_{1\perp},a_1>\frac{1}{2L}\bar{u} (p_{1\perp},a_1) \gamma^+ u(p_{2\perp},a_2) e^{ip_1 \cdot x} e^{-ip_2 \cdot (x+\xi)}<p_{2\perp},a_2|P_{\perp}=0,\alpha_2>\nonumber\\
  &&+\sum_{\alpha_1,\alpha_2,a_{1e},a_{1\gamma},a_{2e}} \int \frac{d^2 p_{1e\perp}}{(2\pi)^2}\int \frac{d^2 p_{1\gamma\perp}}{(2\pi)^2}\int \frac{d^2 p_{2e\perp}}{(2\pi)^2} <P_{\perp}=0,\alpha_1|p_{1e\perp},a_{1e},p_{1\gamma\perp},a_{1\gamma}>\nonumber\\&&\frac{1}{2L}\bar{u}(p_{1e\perp},a_{1e}) \gamma^+ u(p_{2e\perp},a_{2e}) e^{ip_{1e} \cdot x} e^{-ip_{2e} \cdot (x+\xi)}<p_{2e\perp},a_{2e},p_{1\gamma\perp},a_{1\gamma}|P_{\perp}=0,\alpha_2>]\nonumber\\
  &&= \int \frac{d^4x}{(2\pi)^4} \int\frac{d\xi^- d^2 \xi^{\perp}}{(2\pi)^3} \sum_{\alpha_1,\alpha_2,a_1,a_2}\int \frac{d^2 p_{1\perp}}{(2\pi)^2} \int \frac{d^2 p_{2\perp}}{(2\pi)^2} e^{i(k+p_1-p_2)\cdot x} e^{i(k+P_c-p_2)\cdot \xi} \nonumber\\&& \times <P_{\perp}=0,\alpha_1|p_{1\perp},a_1>\frac{1}{2L}\bar{u} (p_{1\perp},a_1) \gamma^+ u(p_{2\perp},a_2)<p_{2\perp},a_2|P_{\perp}=0,\alpha_2>\nonumber\\
  &&+\int \frac{d^4x}{(2\pi)^4} \int\frac{d\xi^- d^2 \xi^{\perp}}{(2\pi)^3}\sum_{\alpha_1,\alpha_2,a_{1e},a_{1\gamma},a_{2e}} \int \frac{d^2 p_{1e\perp}}{(2\pi)^2}\int \frac{d^2 p_{1\gamma\perp}}{(2\pi)^2}\int \frac{d^2 p_{2e\perp}}{(2\pi)^2}e^{i(k+p_1-p_2)\cdot x} e^{i(k+P_c-p_2)\cdot \xi}\nonumber\\&&<P_{\perp}=0,\alpha_1|p_{1e\perp},a_{1e},p_{1\gamma\perp},a_{1\gamma}>\frac{1}{2L}\bar{u}(p_{1e\perp},a_{1e}) \gamma^+ u(p_{2e\perp},a_{2e})<p_{2e\perp},a_{2e},p_{1\gamma\perp},a_{1\gamma}|P_{\perp}=0,\alpha_2>\nonumber\
\end{eqnarray}





\section{Appendix}
\subsection{Appendix A : Convention }

\subsubsection{The Convention of perturbative theory}

In the perturbative theory, we can expand the physical state as:
\begin{eqnarray}
|e\gamma> = \int \frac{dx d^2k^{\perp}}{2(2\pi)^3\sqrt{x(1-x)}} \sum_{\lambda,\lambda_{\gamma}} \Psi^{\Lambda}_{\lambda,\lambda_{\gamma}} |p_e,\lambda;p_{\gamma},\lambda_{\gamma}>\
\end{eqnarray}

where the $\Psi^{\Lambda}_{\lambda,\lambda_{\gamma}}$ is the Wave Function, and it satisfy a normalization condition. Over here, firstly, we need to show the convention:
\begin{align}
\Psi(x) &= \int \frac{dp^+d^2p^{\perp}}{(2\pi)^32p^+} \sum_s (b(p,s)u(p,s) e^{-ip\cdot x} + d^+(p,s) \nu (p,s)e^{ip\cdot x})\\
A_u(x) &= \int \frac{dp^+d^2p^{\perp}}{(2\pi)^32p^+} \sum_s (a(p,s) \epsilon_u(p,s) e^{-ip\cdot x} + a^+(p,s) \epsilon_u(p,s) e^{ip\cdot x})\\
\{b(p,s),b^+(p^{\prime},s^{\prime})\} &= \{d(p,s),d^+(p^{\prime},s^{\prime})\} = 2(2\pi)^3 p^+ \delta^3(p-p^{\prime})\delta^{s}_{s^{\prime}}\\
[a(p,s),a^+(p^{\prime},s^{\prime}) ] &= 2(2\pi)^3 p^+ \delta^3(p-p^{\prime})\delta^{s}_{s^{\prime}}
\end{align}

And we can introduce the convention into the $<e\gamma|e\gamma>$, it's easy to get normalization condition of WF:
\begin{align}
<P,\Lambda|P^{\prime},\Lambda> &= Z + \int \frac{dx_1d^2k_1^{\perp}}{2(2\pi)^3\sqrt{x_1(1-x_1)}} \int \frac{dx_2 d^2k_2^{\perp}}{2(2\pi)^3\sqrt{x_2(1-x_2)}}\sum_{\lambda,\lambda_{\gamma}}\sum_{\lambda^{\prime},\lambda^{\prime}_{\gamma}}\nonumber\\
& <0|b(k_1,\lambda) a(k_{1e},\lambda_{\gamma}) \Psi^{\Lambda,\star}_{\lambda,\lambda_{\gamma}}\Psi^{\Lambda}_{\lambda^{\prime},\lambda^{\prime}_{\gamma}}b^+(k_2,\lambda^{\prime})a^+(k_{2e},\lambda^{\prime}_{\gamma})|0>\nonumber\\
&=Z + \int \frac{dx_1 d^2k_1^{\perp}}{2(2\pi)^3\sqrt{x_1(1-x_1)}} \int \frac{dx_2 d^2k_2^{\perp}}{2(2\pi)^3\sqrt{x_2(1-x_2)}}\sum_{\lambda,\lambda_{\gamma}}\sum_{\lambda^{\prime},\lambda^{\prime}_{\gamma}}\nonumber\\
&<0|\Psi^{\Lambda,\star}_{\lambda,\lambda_{\gamma}}\Psi^{\Lambda,}_{\lambda^{\prime},\lambda^{\prime}_{\gamma}}|0> (2(2\pi)^3)^2 p_e^+ p_{\gamma}^+ \delta^3(p_e-p_e^{\prime}) \delta^3(p_{\gamma}-p_{\gamma}^{\prime}) \delta^{\lambda}_{\lambda^{\prime}} \delta^{\lambda_{\gamma}}_{\lambda_{\gamma}^{\prime}}\nonumber\\
&=Z + \int \frac{dx_1 d^2k_1^{\perp}}{2(2\pi)^3\sqrt{x_1(1-x_1)}} \int \frac{dx_2 d^2k_2^{\perp}}{2(2\pi)^3\sqrt{x_2(1-x_2)}}\sum_{\lambda,\lambda_{\gamma}}\sum_{\lambda^{\prime},\lambda^{\prime}_{\gamma}}\nonumber\\
&<0|\Psi^{\Lambda,\star}_{\lambda,\lambda_{\gamma}}\Psi^{\Lambda,}_{\lambda^{\prime},\lambda^{\prime}_{\gamma}}|0> (2(2\pi)^3)^2 p_e^+ p_{\gamma}^+ \delta^3(p_e-p_e^{\prime}) \delta^3(P-p_e-P^{\prime}+p_e^{\prime}) \delta^{\lambda}_{\lambda^{\prime}} \delta^{\lambda_{\gamma}}_{\lambda_{\gamma}^{\prime}}\nonumber\\
&=Z+\int \frac{dx d^2k^{\perp}}{2(2\pi)^3} \sum_{\lambda,\lambda_{\gamma}} \Psi^{\Lambda,\star}_{\lambda,\lambda_{\gamma}}(p_e,P-p_e)\Psi^{\Lambda,}_{\lambda,\lambda_{\gamma}}(p_e,P^{\prime}-p_e) 2(2\pi)^3P^+\delta^3(P-P^{\prime})\nonumber\
\end{align}

where the P is momentum of the dressed electron. And we can get the WF from the Ref.~\cite{prd3}, is show:
\begin{numcases}{}
  \Psi^{\uparrow}_{+\frac{1}{2} +1}(x,\overrightarrow{k}_{\perp}) = -\sqrt{2}\frac{(-k^1+ik^2)}{x(1-x)}\phi\nonumber\\
  \Psi^{\uparrow}_{+\frac{1}{2} -1}(x,\overrightarrow{k}_{\perp}) = -\sqrt{2}\frac{(+k^1+ik^2)}{1-x}\phi\nonumber\\
  \Psi^{\uparrow}_{-\frac{1}{2} +1}(x,\overrightarrow{k}_{\perp}) = \sqrt{2}(M-\frac{m}{x})\phi\nonumber\\
  \Psi^{\uparrow}_{-\frac{1}{2} -1}(x,\overrightarrow{k}_{\perp}) = 0 \nonumber
\end{numcases}

\begin{numcases}{}
  \Psi^{\downarrow}_{+\frac{1}{2} +1}(x,\overrightarrow{k}_{\perp}) = -\sqrt{2}\frac{(-k^1+ik^2)}{x(1-x)}\phi\nonumber\\
  \Psi^{\downarrow}_{+\frac{1}{2} -1}(x,\overrightarrow{k}_{\perp}) = -\sqrt{2}\frac{(+k^1+ik^2)}{1-x}\phi\nonumber\\
  \Psi^{\downarrow}_{-\frac{1}{2} +1}(x,\overrightarrow{k}_{\perp}) = \sqrt{2}(M-\frac{m}{x})\phi\nonumber\\
  \Psi^{\downarrow}_{-\frac{1}{2} -1}(x,\overrightarrow{k}_{\perp}) = 0 \nonumber
\end{numcases}

\begin{eqnarray}
\phi = \frac{e/\sqrt{1-x}}{M^2-(\overrightarrow{k}^2_{\perp}+m^2)/x-(\overrightarrow{k}^2_{\perp}+\lambda^2)/(1-x)}\nonumber
\end{eqnarray}

Another is worth mentioning, there are definition of the light-cone coordinate and normalization condition which are very different from the BLFQ and are given by:
\begin{eqnarray}
x^+ &=&x^0+x^3\nonumber\\
x^- &=&x^0-x^3\nonumber\\
\bar{u}(p,\lambda)u(p,\lambda^{\prime}) &=& 2m \delta^{\lambda}_{\lambda^{\prime}}
\end{eqnarray}

and we can find a interesting result:
\begin{eqnarray}
  &&g^{\mu\nu} =
  \begin{pmatrix}
  0 & 2 & 0 & 0 \\
  2 & 0 & 0 & 0 \\
  0 & 0 & -1 & 0 \\
  0 & 0 & 0 & -1
\end{pmatrix},\ \ \
g_{\mu\nu} =
\begin{pmatrix}
  0 & \frac{1}{2} & 0 & 0 \\
  \frac{1}{2} & 0 & 0 & 0 \\
  0 & 0 & -1 & 0 \\
  0 & 0 & 0 & -1
\end{pmatrix}
\nonumber\\
&&\int \frac{d \xi^-}{2\pi} e^{-i k^+ \xi^-} = 2\delta(k^+)\nonumber\\
&&\bar{u}(p,\lambda) \gamma^+ u(p, \lambda^{\prime}) = p^+ \delta^{\lambda}_{\lambda^{\prime}}
\end{eqnarray}

\subsubsection{the convention of BLFQ}

According to the BLFQ, we use the definition about the light-front coordinate:
\begin{eqnarray}
  x^+ &=& x^0 + x^3 \nonumber\\
  x^- &=& x^0 - x^3 \nonumber\
\end{eqnarray}

So over here, we can use the same metric,and the same expression of delta function.
\begin{eqnarray}
  &&g^{\mu\nu} =
  \begin{pmatrix}
  0 & 2 & 0 & 0 \\
  2 & 0 & 0 & 0 \\
  0 & 0 & -1 & 0 \\
  0 & 0 & 0 & -1
\end{pmatrix},\ \ \
g_{\mu\nu} =
\begin{pmatrix}
  0 & \frac{1}{2} & 0 & 0 \\
  \frac{1}{2} & 0 & 0 & 0 \\
  0 & 0 & -1 & 0 \\
  0 & 0 & 0 & -1
\end{pmatrix}
\nonumber\\
&&\int \frac{d \xi^-}{2\pi} e^{-i k^+ \xi^-} = 2\delta(k^+)\nonumber\
\end{eqnarray}

But in BLFQ, we use the 2D-HO represent the transverse basis and the plane-wave represent the longitudinal basis. And both the expand of state and normalization condition are all difference. Firstly, we give the process how to introduce the 2D-HO basis:
\begin{eqnarray}
  <P,\Lambda| &=& <P^+,\Lambda|<P^{\perp}|\nonumber\\
  <P^{\prime},\Lambda|P,\Lambda> &=& 2(2\pi)^3 P^+ \delta^3(P^{\prime} - P) \nonumber\\
  \{b(P,\Lambda),b^+(P,\Lambda)\} &=& 2(2\pi)^3 P^+ \delta^3(P^{\prime} - P) \nonumber\
\end{eqnarray}

where the normalization condition and anticommutation are in momentum space. if we want to introduce the 2D-HO basis, we need to expand the momentum state by:
\begin{eqnarray}
  <P^{\perp}| &=& \sum_{n,m} <P^{\perp}|n,m><n,m|\nonumber\\
  <P^{\perp}|n,m> &=& \Phi_{n,m}(P^{\perp})\nonumber\\
  <P^{\perp}| &=& \sum_{n,m} <n,m|\Phi_{n,m}(P^{\perp})\nonumber\\
  \delta(p^{\prime +}-p^+) &=& \frac{L}{2\pi} \delta^{p^{\prime +}}_{p^+}\nonumber\\
  a(P^+,P^{\perp},\Lambda) &=& a_{\alpha} \Phi_{n,m}(P^{\perp}), \ \ \ \alpha=\{n,m,p^+,\lambda\}\nonumber\\
  <P,\Lambda|P^{\prime},\Lambda> &=& (2\pi)^2\delta (P_{\perp}-P_{\perp}^{\prime}) \delta_{P^+}^{P^{\prime +}}\nonumber\
\end{eqnarray}

From the last expression, we can get the anticommutation of the operator:
\begin{eqnarray}
  \{a(P,\Lambda),a^+(P,\Lambda)\} &=& (2\pi)^2 \delta^2(P_{\perp}-P^{\prime}_{\perp}) \delta_{P^+}^{P^{\prime +}} \nonumber\\
  \{a_{\alpha},a^+_{\bar{\alpha}}\} &=& \delta^{\alpha}_{\bar{\alpha}}\nonumber\\
  \sum_{n,m} \Phi_{n,m}(P_{\perp}) \Phi^{\ast}_{n,m}(P^{\prime}_{\perp}) &=& (2\pi)^2 \delta^2(P_{\perp}-P^{\prime}_{\perp})\nonumber\
\end{eqnarray}

It's worth to mention that BLFQ is numerical method. So our every state must normalize to the 1, not delta function. That's why we need to rewrite the convention for our method, and the result is shown:
\begin{eqnarray}
  <P_{\perp},\alpha|P_{\perp},\alpha> &=& 1\nonumber\\
  |P_{\perp},\alpha> &=& \sum_{p^+,\lambda} \int \frac{d^2 p_{\perp}}{(2\pi)^2}|p,\lambda><p,\lambda|P_{\perp},\alpha> \nonumber\\
  &=& \sum_{p^+,\lambda} \int \frac{d^2 p_{\perp}}{(2\pi)^2}|p,\lambda> \Psi(P_{\perp},\alpha;p,\lambda)\nonumber\\
  &=& \sum_{p^+,\lambda} \int \frac{d^2 p_{\perp}}{(2\pi)^2}|p,\lambda> \Phi_{n,m}(p_{\perp})\delta^{p^+}_{p^+_{\alpha}}\delta^{\lambda}_{\lambda_{\alpha}}\nonumber\\
  &=& \int \frac{d^2 p_{\perp}}{(2\pi)^2}|p,\lambda> \Phi_{n,m}(p_{\perp})\label{App2.1}\\
  <p^{\prime},\lambda^{\prime}|p,\lambda> &=& (2\pi)^2\delta^2(p^{\prime}_{\perp}-p_{\perp}) \delta^{p^{\prime +}}_{p^+}\delta^{\lambda^{\prime}}_{\lambda}\nonumber\\
  <P_{\perp},\alpha|P_{\perp},\alpha> &=& \int \frac{d^2 p_{2\perp}}{(2\pi)^2}\int \frac{d^2 p_{1\perp}}{(2\pi)^2} \sum_{p^+_2,\lambda_2}\sum_{p^+_1,\lambda_1} \Psi^{\ast}(P_{\perp},\alpha; p_2,\lambda_2)\Psi(P_{\perp},\alpha; p_1,\lambda_1) <p_2,\lambda_2|p_1,\lambda_1> \nonumber\\
  &=& \int \frac{d^2 p_{2\perp}}{(2\pi)^2}\int \frac{d^2 p_{1\perp}}{(2\pi)^2}  \sum_{p^+_2,\lambda_2}\sum_{p^+_1,\lambda_1}\Psi^{\ast}(P_{\perp},\alpha; p_{2},\lambda)\Psi(P_{\perp},\alpha; p_{1},\lambda) (2\pi)^2\delta^2(p_{2\perp}-p_{1\perp}) \delta^{p^{ +}_2}_{p^+_1}\delta^{\lambda_2}_{\lambda_1} \nonumber\\
  &=& \int \frac{d^2 p_{\perp}}{(2\pi)^2} \Phi^{\ast}_{n,m}(p_{\perp})\Phi_{n,m}(p_{\perp})\nonumber\\ &=& 1 \nonumber\
\end{eqnarray}

where the $<P_{\perp},\alpha|$ is a basis with 2D-HO basis in the transverse direction and $\alpha = \{n,m,k^+,\lambda\}$. In the above derivation, we only consider the singule particle fock sector. But in practice, we need to consider two-body fock sector with the exception of single-particle fock sector. So the normalization condition is given by:
\begin{eqnarray}
  <P_{\perp},\alpha| &=& \int \frac{d^2 p_{\perp}}{(2\pi)^2} \Psi^{\ast}(P_{\perp},\alpha;p,\lambda) <p,\lambda|\nonumber\\&& + \int \frac{d^2 p_{1\perp}}{(2\pi)^2} \int \frac{d^2 p_{2\perp}}{(2\pi)} \Psi^{\ast}(P_{\perp},\alpha;p_{1\perp}\lambda_1,p_{2\perp}\lambda_2) <p_{1\perp}\lambda_1,p_{2\perp}\lambda_2|\nonumber\\
  <P_{\perp}, \alpha|P_{\perp},\alpha> &=& \int \frac{d^2 p_{\perp}}{(2\pi)^2} \Psi^{\ast}(P_{\perp},\alpha; p_{\perp})\Psi(P_{\perp},\alpha; p_{\perp}) \nonumber\\&&+ \int \frac{d^2 p_{1\perp}}{(2\pi)^2} \int \frac{d^2 p_{2\perp}}{(2\pi)} \Psi(P_{\perp},\alpha;p_{1\perp}\lambda_1,p_{2\perp}\lambda_2)\Psi^{\ast}(P_{\perp},\alpha;p_{1\perp}\lambda_1,p_{2\perp}\lambda_2)\nonumber\\&=& 1\nonumber\
\end{eqnarray}

As like the expression of Eq.(\ref{App2.1}), we can give the form of WF:
\begin{eqnarray}
  \Psi(P_{\perp},\alpha;p_e,\lambda_e,p_{\gamma},\lambda_{\gamma}) = \Phi_{n_e,m_e}(p_{e\perp}) \Phi_{n_{\gamma},m_{\gamma}}(p_{\gamma\perp})\delta^{a_e}_{a_{e\alpha}}\delta^{a_{\gamma}}_{a_{\gamma\alpha}},\ \ \ a=\{p^+,\lambda\}
\end{eqnarray}

And we also need to change the expandion of the field;
\begin{eqnarray}
  &&\Psi(\xi) = \sum_{j,\lambda} \frac{1}{\sqrt{2L}} \int \frac{d^2 p_{\perp}}{(2\pi)^2} [b(p_{\perp},a)u(p_{\perp},a)e^{-ip\cdot \xi}-d^{\dag}(p_{\perp},a)v(p_{\perp},a)e^{ip\cdot \xi}] \nonumber\\
  &&\bar{u}(p_{\perp},a) \gamma^+ u(p_{\perp},a) = \delta^{\lambda^{\prime}}_{\lambda} \nonumber\
\end{eqnarray}

\subsection{Appendix B: Discrete symmetries on the light front}

\subsubsection{Parity symmetry}

To implement the light-front parity operation $P_{\perp}$, they defined the spatial components of any vector and the transformation as:
\begin{eqnarray}
d^{\mu} &=& (d^+,d^-,d^L,d^R)\nonumber\\
d^+ = d^0 + d^3
&& d^- = d^0 - d^3\nonumber\\
d^R = d^1 - i d^2
&& d^L = d^1 + i d^2\nonumber\\
d^+ \xrightarrow{p_{\perp}} d^+
&& d^- \xrightarrow{p_{\perp}} d^- \nonumber\\
d^R \xrightarrow{p_{\perp}} -d^R
&& d^L \xrightarrow{p_{\perp}} -d^L\nonumber\
\end{eqnarray}

notes that this transformation is equivalent to letting $d^1 \xrightarrow{p_{\perp}} -d^1$ with all other components transforming into themselves. Considering the commutator $[x_i , p_j] = i\delta_{ij}$ and $L=r \times p$, we find that $P_{\perp}$ flips the spin as well.
We thus act the parity transformation at the operator level via:
\begin{eqnarray}
P_{\perp} a^{\lambda}_{p^L,p^R} P^+_{\perp} &=& \eta_a a^{-\lambda}_{-p^R,-p^L} \nonumber\\
P_{\perp} b^{\lambda}_{p^L,p^R} P^+_{\perp} &=& \eta_b b^{-\lambda}_{-p^R,-p^L} \
\end{eqnarray}

and if let the operator act on the field operator, we can get:
\begin{eqnarray}
P_{\perp} \Psi(x) P_{\perp} &=& \int \frac{dk^+d^2 k_{\perp}}{\sqrt{2k^+(2\pi)^3}}P_{\perp}[a^{\lambda}_{p^L,p^R}u(k,\lambda)e^{-ik\cdot x} +
b^{\dag\lambda}_{p^L,p^R}v(k,\lambda)e^{ik\cdot x}]P^+_{\perp}\nonumber\\
&=& \int \frac{dk^+d^2 k_{\perp}}{\sqrt{2k^+(2\pi)^3}}[a^{-\lambda}_{-p^L,-p^R}u(k,\lambda)e^{-ik\cdot x} +b^{\dag-\lambda}_{-p^L,-p^R}v(k,\lambda)e^{ik\cdot x}]\nonumber\\
&=& \int \frac{d\tilde{k}^+d^2 \tilde{k}_{\perp}}{\sqrt{2\tilde{k}^+(2\pi)^3}}[\eta_aa^{-\lambda}_{\tilde{p}^L,\tilde{p}^R}\gamma^1\gamma^5u(\tilde{k},-\lambda)e^{-i\tilde{k}\cdot \tilde{x}} +\eta^{\ast}_bb^{\dag-\lambda}_{\tilde{p}^L,\tilde{p}^R}\gamma^1\gamma^5v(\tilde{k},-\lambda)e^{i\tilde{k}\cdot \tilde{x}}]\nonumber\\
&=& \eta_a \gamma^1\gamma^5 \Psi(\tilde{x})\
\end{eqnarray}

where we defined $\tilde{d}=(d^+,d^-,-d^R,-d^L)$ and $\eta^{\ast}_b=\eta_a,\eta^{\ast}_a\eta_a = 1$. And we also use a result:
\begin{eqnarray}
u(k,\lambda) &=& (\sqrt{p\cdot \sigma} \xi , \sqrt{p\cdot \bar{\sigma}} \xi)\nonumber\\
\sigma^{\mu} p\cdot \sigma &=& p\cdot \bar{\sigma} \sigma^{\mu},\ \ \ \mu \neq 1\nonumber\\
u(k,\lambda) &=& \gamma^1\gamma^5 u(\tilde{k},-\lambda)\
\end{eqnarray}

And then, we can get:
\begin{eqnarray}
|P,\Lambda>= P_{\perp} |-P^1,P^2,P^+,P^-,-\Lambda>\
\end{eqnarray}

%\subsubsection{Time reversal}



























\begin{thebibliography}{90}
%
\bibitem{prd1} arxiv: hep-ph 1508.06964
\bibitem{prd2} Physics Letters B 578(2004) 109
\bibitem{prd3} hep-th/0003082





\end{thebibliography}
\end{document}
